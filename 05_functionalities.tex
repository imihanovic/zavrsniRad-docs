\subsection{Funkcionalnosti aplikacije}

\begin{figure}[H]
	\includegraphics[width=1\linewidth,clip=]{assets/loginForma.png}
	\centering
	\caption{Login forma}
	\label{fig:loginForm}
\end{figure}

Na slici~\ref{fig:loginForm} prikazana je forma za prijavu korisnika dobivena pomoću Scaffold Identitya. Nakon uspješnog unosa korisničkog imena i lozinke, korisnik se autenticira u sustav ili mu se odbija pristup u slučaju krivo unesenih podataka. Uspješnom prijavom autenticirani korisnik ima pristup resursima aplikacije za koje njegova korisnička uloga ima postavljena autorizacijska prava. Korisnici s ulogom \texttt{Administrator} i \texttt{SysAdministrator} odmah nakon uspješne autentikacije preusmjeravaju se na glavni aplikacijski izbornik. Autenticirani korisnici s ulogom \texttt{Doctor} i \texttt{Nurse} preusmjeravaju se na formu za odabir smjene te nemaju pravo korištenja aplikacije dok ne ispune taj međukorak. Provjera poveznice doktora, medicinske sestre i pripadajuće smjene implementirana je kolačićima.

\begin{figure}[H]
	\includegraphics[width=0.6\linewidth,clip=]{assets/formaShift.png}
	\centering
	\caption{Forma za odabir smjene}
	\label{fig:shiftForm}
\end{figure}

Nakon prijave u sustav, korisnici su ulogom \texttt{Doctor} ili \texttt{Nurse} preusmjereni su na stranicu za odabir smjene. Ako je za trenutnog korisnika već odabrana smjena, bit će preusmjeren na glavni izbornik. Nakon odabrane smjene, 

\begin{figure}[H]
	\includegraphics[width=1\linewidth,clip=]{assets/sysAdminIndex.png}
	\centering
	\caption{Početna stranica aplikacije}
	\label{fig:homePage}
\end{figure}

Na slici~\ref{fig:homePage} prikazana je početna stranica aplikacije s izbornikom s lijeve strane za korisnika s ulogom \texttt{SysAdministrator}. Broj izbora na spomenutom izborniku ovisi o ulozi korisnika.

\begin{figure}[H]
	\includegraphics[width=1\linewidth,clip=]{assets/viewAllUsers.png}
	\centering
	\caption{Prikaz svih korisnika aplikacije}
	\label{fig:allUsers}
\end{figure}

Korisnicima s ulogom \texttt{Administrator} ili \texttt{SysAdministrator} omogućen je prikaz svih korisnika aplikacije. Na stranicu su implementirane mogućnosti pretrage, filtriranja i sortiranja zapisa. Za svakog korisnika postoji i botun s kojim je omogućeno uređivanje podataka i brisanje pacijenta iz sustava.

\begin{figure}[H]
	\includegraphics[width=1\linewidth,clip=]{assets/newPatient.png}
	\centering
	\caption{Unos novog pacijenta}
	\label{fig:newPatient}
\end{figure}

Slika~\ref{fig:newPatient}prikazuje formu za unos novog pacijenta u sustav primarne zdravstvene zaštite. Prilikom kreiranja novog pacijenta u sustav, unose se svi potrebni podatci. Najvažniji podatci su \texttt{OIB} i \texttt{MBO} koji jedinstveno identificiraju pacijenta. \texttt{OIB} i \texttt{MBO} su tekstualna \textit{string} polja kako bi se omogućilo pravilno spremanje vrijednosti u bazu podataka ako polje počinje s nulom. U svrhu zaštite integriteta podataka ograničen je unos samo znamenki te se \texttt{OIB} sastoji od jedanaest, a \texttt{MBO} od devet znamenki. Sva polja za unos su obavezna osim odabira određenog doktora. 

\begin{figure}[H]
	\includegraphics[width=1\linewidth,clip=]{assets/newCheckup.png}
	\centering
	\caption{Forma za unos novog \texttt{Checkup} entiteta}
	\label{fig:newCheckup}
\end{figure}

Na slici~\ref{fig:newCheckup} prikazana je forma za unos. Za navedenu stranicu prikazani su ispisi~\ref{subsubsec:.cshtml} za \texttt{.cshtml} i ispis~\ref{subsubsec:.cshtml.cs} za \texttt{.cshtml.cs} datoteke. Nakon spremanja, korisnik je preusmjeren na stranicu prikazanu na slici~\ref{fig:allCheckups}.

\begin{figure}[H]
	\includegraphics[width=1\linewidth,clip=]{assets/allCheckups.png}
	\centering
	\caption{Prikaz svih elemenata entiteta \texttt{Checkup}}
	\label{fig:allCheckups}
\end{figure}

\begin{figure}[H]
	\includegraphics[width=1\linewidth,clip=]{assets/newAppointment.png}
	\centering
	\caption{Unos novog pregleda u čekaonicu}
	\label{fig:newAppointment}
\end{figure}

Na slici~\ref{fig:newAppointment} prikazan je unos novog pregleda za koji se odabire pacijent koristeći select2 i razlog dolaska. Nakon unosa, korisnik je preusmjeren na stranicu koja predstavlja čekaonicu \texttt{WaitingRoom} ili na prikaz svih pregleda ovisno o ulozi.

\begin{figure}[H]
	\includegraphics[width=1\linewidth,clip=]{assets/waitingRoom.png}
	\centering
	\caption{Čekaonica}
	\label{fig:waitingRoom}
\end{figure}

Slika~\ref{fig:waitingRoom} prikazuje čekaonicu za korisnike s ulogama \texttt{Doctor} i \texttt{Nurse}. Pojedini status pregleda, označen je bojom.

\begin{figure}[H]
	\includegraphics[width=1\linewidth,clip=]{assets/appointmentDetails.png}
	\centering
	\caption{Detalji pregleda}
	\label{fig:appointmentDetails}
\end{figure}

Pritiskom na botun za detalje pregleda, otvara se stranica prikazana na slici~\ref{fig:appointmentDetails} na kojoj je omogućen unos novih recepata za lijek, cjepiva i izvješća. Pritiskom na botun \texttt{End appointment}, status pregleda mijenja se na \texttt{Done} i tada su izmjene dostupne samo korisniku s ulogom \texttt{SysAdministrator}.