\subsection{Funkcionalnosti aplikacije}

\begin{figure}[H]
	\includegraphics[width=0.5\linewidth,clip=]{assets/loginForma.png}
	\centering
	\caption{Forma za prijavu}
	\label{fig:loginForm}
\end{figure}

Na slici~\ref{fig:loginForm} prikazana je forma za prijavu korisnika dobivena pomoću Scaffold Identityja. Nakon uspješnog unosa korisničkog imena i lozinke, korisnik se autenticira u sustav ili mu se odbija pristup u slučaju pogrešno unesenih podataka. Korisnici s ulogom \texttt{Administrator} i \texttt{SysAdministrator}, nakon uspješne autentikacije, preusmjeravaju se na glavni aplikacijski izbornik. Autenticirani korisnici s ulogom \texttt{Doctor} ili \texttt{Nurse} preusmjeravaju se na formu za odabir smjene te nemaju pravo korištenja aplikacije dok ne ispune taj međukorak. Provjera poveznice doktora, medicinske sestre i pripadajuće smjene implementirana je kolačićima (engl. \textit{cookies}).

\begin{figure}[H]
	\includegraphics[width=0.6\linewidth,clip=]{assets/formaShift.png}
	\centering
	\caption{Forma za odabir smjene}
	\label{fig:shiftForm}
\end{figure}

Nakon prijave u sustav, korisnici s ulogom \texttt{Doctor} ili \texttt{Nurse} preusmjereni su na stranicu za odabir smjene. Ako je za trenutnog korisnika već odabrana smjena, bit će preusmjeren na glavni izbornik. Nakon odabrane smjene, podatci o novoj smjeni spremljeni su u kolačiće i korišteni za prikaz na izborniku što je vidljivo na slici~\ref{fig:newAppointment}.

\begin{figure}[H]
	\includegraphics[width=1\linewidth,clip=]{assets/sysAdminIndex.png}
	\centering
	\caption{Početna stranica aplikacije}
	\label{fig:homePage}
\end{figure}

Na slici~\ref{fig:homePage} prikazana je početna stranica aplikacije za korisnika s ulogom \\\texttt{SysAdministrator} s izbornikom s lijeve strane. Broj izbora na spomenutom izborniku ovisi o ulozi korisnika.

\begin{figure}[H]
	\includegraphics[width=1\linewidth,clip=]{assets/viewAllUsers.png}
	\centering
	\caption{Prikaz svih korisnika aplikacije}
	\label{fig:allUsers}
\end{figure}

Korisnicima s ulogom \texttt{Administrator} ili \texttt{SysAdministrator} omogućen je prikaz svih korisnika aplikacije. Na stranicu su implementirane mogućnosti pretrage, filtriranja i sortiranja zapisa. Za svakog korisnika postoji i botun s kojim je omogućeno uređivanje podataka i brisanje pacijenta iz sustava.

\begin{figure}[H]
	\includegraphics[width=1\linewidth,clip=]{assets/newPatient.png}
	\centering
	\caption{Unos novog pacijenta}
	\label{fig:newPatient}
\end{figure}

Slika~\ref{fig:newPatient} prikazuje formu za unos novog pacijenta u sustav. Prilikom kreiranja novog pacijenta u sustav unose se svi potrebni podatci. Najvažniji su podatci \texttt{OIB} i \texttt{MBO} koji jedinstveno identificiraju pacijenta. \texttt{OIB} i \texttt{MBO} tekstualna su \textit{string} polja kako bi se omogućilo pravilno spremanje vrijednosti u bazu podataka ako polje počinje s nulom. U svrhu zaštite integriteta podataka ograničen je unos samo znamenki te se \texttt{OIB} sastoji od jedanaest, a \texttt{MBO} od devet znamenki. Sva su polja za unos obavezna osim odabira određenog liječnika. 

\begin{figure}[H]
	\includegraphics[width=1\linewidth,clip=]{assets/PatientDetails.png}
	\centering
	\caption{Informacije o pacijentu}
	\label{fig:patientDetails}
\end{figure}

Slika~\ref{fig:patientDetails} prikazuje informacije o pacijentu. Klikom na poveznice moguće je vidjeti sve preglede, recepte, cjepiva i dolaske pacijenta. Omogućeno je i uređivanje podataka pacijenta.

\begin{figure}[H]
	\includegraphics[width=1\linewidth,clip=]{assets/newCheckup.png}
	\centering
	\caption{Forma za unos novog elementa entiteta \texttt{Checkup}}
	\label{fig:newCheckup}
\end{figure}

Na slici~\ref{fig:newCheckup} prikazana je forma za unos. Za navedenu stranicu prikazani su ispisi~\ref{subsubsec:.cshtml} za datoteku \texttt{.cshtml} i ispis~\ref{subsubsec:.cshtml.cs} za datoteku \texttt{.cshtml.cs}. Nakon spremanja, korisnik je preusmjeren na stranicu prikazanu na slici~\ref{fig:allCheckups}.

\begin{figure}[H]
	\includegraphics[width=1\linewidth,clip=]{assets/allCheckups.png}
	\centering
	\caption{Prikaz svih elemenata entiteta \texttt{Checkup}}
	\label{fig:allCheckups}
\end{figure}

\begin{figure}[H]
	\includegraphics[width=1\linewidth,clip=]{assets/newAppointment.png}
	\centering
	\caption{Unos novog pregleda u čekaonicu}
	\label{fig:newAppointment}
\end{figure}

Na slici~\ref{fig:newAppointment} prikazan je unos novog pregleda u čekaonicu za koji se odabire pacijent koristeći padajući izbornik \texttt{Select2} i unosi se razlog dolaska. Nakon unosa, korisnik je preusmjeren na stranicu koja predstavlja čekaonicu \texttt{WaitingRoom} ili na prikaz svih pregleda (ovisno o ulozi).

\begin{figure}[H]
	\includegraphics[width=1\linewidth,clip=]{assets/waitingRoom.png}
	\centering
	\caption{Čekaonica}
	\label{fig:waitingRoom}
\end{figure}

Slika~\ref{fig:waitingRoom} prikazuje čekaonicu za korisnike s ulogama \texttt{Doctor} i \texttt{Nurse}. Pojedini status pregleda, označen je zasebnom bojom.

\begin{figure}[H]
	\includegraphics[width=1\linewidth,clip=]{assets/appointmentDetails.png}
	\centering
	\caption{Detalji pregleda}
	\label{fig:appointmentDetails}
\end{figure}

Pritiskom na botun za detalje pregleda, otvara se stranica prikazana na slici~\ref{fig:appointmentDetails} na kojoj je omogućen unos novih recepata za lijek, cjepiva i izvješća. Pritiskom na botun \texttt{End appointment}, status pregleda mijenja se na \texttt{Done} i tada su izmjene dostupne samo korisniku s ulogom \texttt{SysAdministrator}. Na slikama~\ref{fig:newCheckupAppointment}, ~\ref{fig:dateCheckupAppointment} i~\ref{fig:timeCheckupAppointment} prikazan je postupak narudžbe pacijenta na pregled u bolnici.

\begin{figure}[H]
	\includegraphics[width=0.6\linewidth,clip=]{assets/newCheckupAppointment.png}
	\centering
	\caption{Odabir pregleda}
	\label{fig:newCheckupAppointment}
\end{figure}

Na slici~\ref{fig:newCheckupAppointment} prikazan je početak narudžbe pacijenta na pregled u kojem liječnik odabire pregled pomoću padajućeg izbornika \texttt{Select2}.

\begin{figure}[H]
	\includegraphics[width=0.6\linewidth,clip=]{assets/dateCheckupAppointment.png}
	\centering
	\caption{Odabir datuma pregleda}
	\label{fig:dateCheckupAppointment}
\end{figure}

Nakon odabira pregleda, liječnik odabire datum pregleda te je preusmjeren na stranicu prikazanu na slici~\ref{fig:timeCheckupAppointment} za odabir termina pregleda.

\begin{figure}[H]
	\includegraphics[width=0.6\linewidth,clip=]{assets/timeCheckupAppointment.png}
	\centering
	\caption{Odabir termina pregleda}
	\label{fig:timeCheckupAppointment}
\end{figure}

Prikazani su samo dostupni termini pregleda koji su generirani za pojedini pregled na temelju njegova trajanja. Za svaki termin, moguće je napraviti tri narudžbe. Kada su tri narudžbe za isti pregled u isto vrijeme kreirane sa statusom \texttt{Active}, navedeni termin ne prikazuje se u padajućem izborniku. Otkazivanje pregleda prikazano je na slici~\ref{fig:cancelCheckup}

\begin{figure}[H]
	\includegraphics[width=0.6\linewidth,clip=]{assets/cancelCheckup.png}
	\centering
	\caption{Otkazivanje pregleda}
	\label{fig:cancelCheckup}
\end{figure}

Na slici~\ref{fig:mailHog} prikazana je poruka poslana pomoću alata \texttt{MailHog} koji omogućava pregled i simulaciju slanja elektroničke pošte. Poruka se kreira prilikom spremanja nove narudžbe na pregled.

\begin{figure}[H]
	\includegraphics[width=1\linewidth,clip=]{assets/mailHog.png}
	\centering
	\caption{Poruka kreirana pomoću alata \texttt{MailHog}}
	\label{fig:mailHog}
\end{figure}