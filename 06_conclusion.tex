\section{Zaključak}

Rezultat ovog završnog rada je web aplikacija za upravljanje podatcima u primarnoj zdravstvenoj zaštiti s naglaskom na jednostavnost i pristupačnost korisnicima s manje računalnog obrazovanja. Aplikacija omogućuje liječnicima i medicinskim sestrama unos pregleda, pacijenata, recepata za lijek i cjepiva, uz dodatne funkcionalnosti za doktora kao što je unos izvješća o pregledu i narudžba na pregled u bolnici. Korisnik s ulogom Administrator zadužen je za unos novih cjepiva, lijekova, pregleda u bolnici i ureda ali nema pristup popisu pacijenata dok su korisniku s ulogom SysAdministrator dostupne sve funkcionalnosti.

U radu je korišten velik broj različitih tehnologija u svrhu učenja istih. Svaka od opisanih tehnologija pruža jednostavnost i različite mogućnosti prilikom izrade web aplikacije. Okvir ASP.NET Core zbog svojih značajki kao što su ugrađene zaštite od sigurnosnih prijetnji, rad aplikacija na različitim operacijskim sustavima, brža obrada zahtjeva i snažna podrška za asinkrono programiranje postaje jedan od najkorištenijih okvira u svijetu programiranja.

S obzirom na to da primarna zdravstvena zaštita obuhvaća velik spektar raznolikih podataka o pacijentima, web aplikacija \textit{PrimeCareMed} ima potencijal za daljnje unapređenje i proširenje. Zdravstvo je širok pojam pa postoje mnoge funkcionalnosti koje mogu biti dodane da se osigura maksimalna iskoristivost sustava. Aplikaciju bi bilo korisno nadograditi statističkim podatcima koji bi se iskoristili za bolji rad sustava, naprimjer broj otkazanih pregleda ili onih na koji pacijent nije došao. Također, potrebna je integracija s drugim sustavima da bi aplikacija bila u potpunosti upotrebljiva.


