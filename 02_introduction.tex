\section{Uvod}

Uzevši u obzir napredovanje računarske tehnologije, sigurno je za reći da postoje aplikacije i sustavi koji su za današnje mogućnosti "zastarjele". Moderna poboljšana tehnologija trebala bi biti dostupna u svim područjima da bi olakšala svakodnevni rad. Prilikom izrade sustava za upravljanje podataka u zdravstvenim područjima potrebno je omogućiti sustav koji je pristupačan i jednostavan za korištenje te uzeti u obzir korištenje aplikacije od strane onih korisnika koji nisu računalno obrazovani. U svrhu toga izrađena je web aplikacija \textit{PrimeCareMed}. 

Pristup aplikaciji imaju četiri vrste korisnika o čijoj ulozi ovise funkcionalnosti koje mogu raditi: \texttt{SysAdministrator}, \texttt{Administrator}, \texttt{Doctor} i \texttt{Nurse}. Liječnik i medicinska sestra unose nove preglede u čekaonicu, imaju uvid u detalje samog pregleda te mogućnost dodavanja cijepljenja i recepata za lijek. Uz navedene zajedničke funkcionalnosti, liječnik ima mogućnost stvaranja narudžbe pacijenta na pregled te odabir termina za isti. Također, imaju uvid u popis pacijenata te dodavanje novog uz uvjet da su matični broj osigurane osobe i osobni identifikacijski broj jedinstveni. SysAdministrator ima omogućene sve funkcionalnosti, dok sam administrator ima mogućnosti unosa novih liječnika, medicinskih sestara ili tehničara, lijekova, cijepiva te samih ureda bez uvida u popis pacijenata i pregleda.

U slijedećim poglavljima bit će predstavljene korištene tehnologije, a zatim će se detaljno i pregledno opisati način na koji je sama aplikacija implementirana.
